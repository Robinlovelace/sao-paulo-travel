\documentclass[]{article}
\usepackage{lmodern}
\usepackage{amssymb,amsmath}
\usepackage{ifxetex,ifluatex}
\usepackage{fixltx2e} % provides \textsubscript
\ifnum 0\ifxetex 1\fi\ifluatex 1\fi=0 % if pdftex
  \usepackage[T1]{fontenc}
  \usepackage[utf8]{inputenc}
\else % if luatex or xelatex
  \ifxetex
    \usepackage{mathspec}
    \usepackage{xltxtra,xunicode}
  \else
    \usepackage{fontspec}
  \fi
  \defaultfontfeatures{Mapping=tex-text,Scale=MatchLowercase}
  \newcommand{\euro}{€}
\fi
% use upquote if available, for straight quotes in verbatim environments
\IfFileExists{upquote.sty}{\usepackage{upquote}}{}
% use microtype if available
\IfFileExists{microtype.sty}{%
\usepackage{microtype}
\UseMicrotypeSet[protrusion]{basicmath} % disable protrusion for tt fonts
}{}
\usepackage[margin=1in]{geometry}
\ifxetex
  \usepackage[setpagesize=false, % page size defined by xetex
              unicode=false, % unicode breaks when used with xetex
              xetex]{hyperref}
\else
  \usepackage[unicode=true]{hyperref}
\fi
\hypersetup{breaklinks=true,
            bookmarks=true,
            pdfauthor={Ana I. Moreno-Monroy (Universitat Rovira i Virgili), Frederico R. Ramos (Fundação Getulio Vargas) and Robin Lovelace (University of Leeds)},
            pdftitle={The unequal distribution of access to education and transport: insights from for São Paulo},
            colorlinks=true,
            citecolor=blue,
            urlcolor=blue,
            linkcolor=magenta,
            pdfborder={0 0 0}}
\urlstyle{same}  % don't use monospace font for urls
\usepackage{graphicx,grffile}
\makeatletter
\def\maxwidth{\ifdim\Gin@nat@width>\linewidth\linewidth\else\Gin@nat@width\fi}
\def\maxheight{\ifdim\Gin@nat@height>\textheight\textheight\else\Gin@nat@height\fi}
\makeatother
% Scale images if necessary, so that they will not overflow the page
% margins by default, and it is still possible to overwrite the defaults
% using explicit options in \includegraphics[width, height, ...]{}
\setkeys{Gin}{width=\maxwidth,height=\maxheight,keepaspectratio}
\setlength{\parindent}{0pt}
\setlength{\parskip}{6pt plus 2pt minus 1pt}
\setlength{\emergencystretch}{3em}  % prevent overfull lines
\providecommand{\tightlist}{%
  \setlength{\itemsep}{0pt}\setlength{\parskip}{0pt}}
\setcounter{secnumdepth}{0}

%%% Use protect on footnotes to avoid problems with footnotes in titles
\let\rmarkdownfootnote\footnote%
\def\footnote{\protect\rmarkdownfootnote}

%%% Change title format to be more compact
\usepackage{titling}

% Create subtitle command for use in maketitle
\newcommand{\subtitle}[1]{
  \posttitle{
    \begin{center}\large#1\end{center}
    }
}

\setlength{\droptitle}{-2em}
  \title{The unequal distribution of access to education and transport: insights
from for São Paulo}
  \pretitle{\vspace{\droptitle}\centering\huge}
  \posttitle{\par}
  \author{Ana I. Moreno-Monroy (Universitat Rovira i Virgili), Frederico R. Ramos
(Fundação Getulio Vargas) and Robin Lovelace (University of Leeds)}
  \preauthor{\centering\large\emph}
  \postauthor{\par}
  \date{}
  \predate{}\postdate{}


% Redefines (sub)paragraphs to behave more like sections
\ifx\paragraph\undefined\else
\let\oldparagraph\paragraph
\renewcommand{\paragraph}[1]{\oldparagraph{#1}\mbox{}}
\fi
\ifx\subparagraph\undefined\else
\let\oldsubparagraph\subparagraph
\renewcommand{\subparagraph}[1]{\oldsubparagraph{#1}\mbox{}}
\fi

\begin{document}
\maketitle

\section{Abstract}\label{abstract}

In many large Latin American cities such as São Paulo, growing social
and economic inequalities are embedded through unfair education and
transport systems. Good schools are mostly concentrated in wealthy
areas, while transport links to school are deficient in deprived areas,
exacerbating the issue. Inequalities in educational and transport
infrastructure are mutually reinforcing: the right to mobility is
intrinsically linked to the right to education. This is manifested by
the overlap between recent protests against unwanted changes to public
education and the social movements contesting increases in public
transport fares. Another manifestation is to be found in the concept of
school accessibility. This paper sheds light on the transport-education
inequality nexus with reference to a new school accessibility measure
applied São Paulo. By capturing both the unequal distribution of schools
and transport services across space, the index allows embedded
inequalities to be better understood and, with political will,
contested. Our index combines information on the spatial distribution of
children and adolescents by income categories, the location of existing
schools, the travel patterns of students, and the travel infrastructure
serving the school catchment area into a single measure. The index is
used to measure school accessibility locally across São Paulo, using
data sources from Population and School Censuses, commonly available in
Latin American cities. The results illustrate how existing inequalities
are amplified by variable accessibility to schools across income groups
and geographical space. We conclude that extending the concept of local
accessibility indicators to education can help to both contest and
constructively tackle embedded social inequalities.

\section{Introduction}\label{introduction}

\textbf{Contents:}

\begin{itemize}
\tightlist
\item
  Motivations
\item
  General measures of accessibility
\item
  School accessibility
\item
  Literature review
\end{itemize}

References transport and social exclusion (Hernandez and Titheridge
2015) and (Hernandez and Titheridge 2015) provides a general framework
to undertand transport-related social exclusion in emerging economies

Evidence on access inequalities in the Brazilian context (Maia et al.
2016)

Active travel children and adolescents in the SPMR (Sá et al. 2015)
\ldots{}

Travel to school is an everyday reality for millions of young people
around the world. The mode, duration, safety, comfort and pollution
levels of this trip has huge impacts on the future generation, yet has
received relatively little academic attention.

Travel options are vital for ensuring a more equitable supply of
educational opportunity to diverse groups. Conversely, poor
accessibility to deprived area can reinforce social inequalities, with
long-term implications. Based on this emerging context, this paper
develops a school accessibility index for local areas.

The first well-known attempt to define accessibility quantitatively was
by Ingram (1971), which presented a range of measures related to
distance (Euclidean and network), barriers and different functions
representing distance decay.

This early work made the distinction between accessibility indeces that
apply to zones or single `desire lines': ``relative accessibility is
defined as a measure of the effort of overcoming spatial separation
between two points, while the integral accessibility is defined as a
measure of the effort of overcoming spatial separation between a point
and all other points within an area'' (Allen, Liu, and Singer 1993).

\section{Area of study}\label{area-of-study}

The São Paulo Metropolitan Region (SPMR) is a large metropolis located
in the South West of Brazil. It extends for over 7,700 squared
kilometers and groups 39 different minucipalities. The estimated
population in 2010 was close to 21 million. \ldots{} ADD MAP, etc.

\section{Data}\label{data}

For building our school accessibility index, we use data from three
different sources. The first one is the 2008 School Census of the
Brazilian National Institute of Educational Studies and Surveys (INEP),
which provides information on the (universe) of public educational
institutions, including their exact location (postal address) and the
number of students enrolled in primary and secondary education. A
clarification about the educational system in Brazil is in place here.
Primary education \textit{ensino fundamental} comprises nine years,
divided on a first year of basic literacy
\textit{clase de alfabetiza\c{c}\~{a}}, followed by eight `` series''.
Primary education is mandatory for children aged 6-14. Secondary
education lasts three years. In principle, the age range of secondary
education students is 15-18. For each school, have geo-localized each
postal adress for a total of XX public primary and XX secondary schools
in the SPRM.

Our second source is information the 2010 Population Census of Brazil at
the \textit{clase de alfabetiza\c{c}\~{a}}, compiled and freely
distributed by the Brazilian Institute of Statistics (IBGE). This
spatial unit is equivalent to enumeration areas defined for surveying
purposes, and contain on average 400 households. There is a total of
30,815 areas in the SPMR. IBGE also provides the digital networks
containing the boundaries of the enumeration areas. From the census
microdata, we have data on the number of inhabitants by age for each
enumeration area. As the data was compiled in 2010 and we are interested
in the number of children in primary school age (ages 6-14), and
adolescents in secondary school age (ages 15-18), we calculate the count
of children in the age range 8-16 (who were 6 and 14 in 2008) and
adolescents in the age range 17-20 (aged 15-18 in 2008). The underlying
assumption is that the location of children and adolescents did not
change dramatically or systematically between 2008 and 2010, which is a
reasonable assumption given thatr residential mobility is relatively
low. using this procedure, we obtain a total number of children in
primary school age of 2,731,999 (approximately 13 percent of the
population) and a total of 1,730,252 tenneagers in secondary school age
(approximately 8 percent of the population).

Out last source is the 2007 Origin-Destination Household Travel Survey
(O-D Survey), carried out by the São Paulo Metropolitan Transport
Authority M\^{}\{e\}tro. we have information at the travel flow-level.
geographic coordinates are provided for the origin and destination of
each trip, as well as additional information such as trip purpose
(includng education), trip duration, number of transfers and mode. The
O-D Survey also provides the exact location of schools. The survey also
provides relevant information on individuals undertaking the trips, such
as their age, income range and educational status (including student).
We focus on trips made by primary and secondary schools with an
indicated motive ``education'' at the destination. The total number of
trips (excluding missing values) was 11,845. Of these, 5.5 percent were
multimodal. In those cases, we assing the mode of the trip leg with the
largest duration. We add all trip leg durations (in minutes) to obtain
the total trip duration.

\section{Method: measuring school
accessibility}\label{method-measuring-school-accessibility}

We calculate three different accessibility measures for different modes:
cumulative, gravitiy-based, and competitive.

The cumulative-opportunity measure for mode \(M\) (Boisjoly G. 2016) is
defined as:

\[ CO_{i}^M= \sum_{j=1}^n(S_{j}f(C_{ij}) \]

\[ f(C_{ij}) = \left\{ 
                \begin{array}{ll}
                  1 if C_{ij}<=t\\
                  0 if C_{ij}<t
                \end{array}
              \right.
  \]

This measure counts the number of schools available from one area within
a certain travel time by mode \(M\) threshold. \(C_{ij}\) is the travel
cost (measured in time) between the centroid of zone i and the centroid
of zone j.

Gravity-based accessibility measure

In spite of its simplicity, Boisjoly and El-Geneidy (2016) have found
that the cumulative-opportunity measure (using a 45 minute threshold) is
highly correlated with a gravity-based accessibility measure
(({\textbf{???}}), ({\textbf{???}})), defined as:

\[ A_{i}^M= \sum_{j=1}^n(S_{j}e^{\beta*C_{ij}}) \]

Where \(A_{i}^M\) is the gravity accessibility at the centroid of area
\(i\) to all schools at area \(j\) using mode \(M\), and \(\beta\) is a
negative exponential cost function. The cost function, based on a
negative exponential decay curve, includes reported number of school
trips in the 2007 Origin-Destination Survey. (Here we can also use
information from the school census on number of students per school to
know how many students the area attracts). The proposed accessibility
measure captures the fact that more proximate schools (S) are more
attractive than those located further away.

Competitive accessibility measure

The measures proposed above take into account the sptail distribution of
schooling opportunities, but not the local demand for schooling. This is
particularly relevant for our equity analysis, since it could be the
case that higher areas are disproportionally served with respect to the
number of potentail students living within a certain travel distance,
whereas the opposite holds for lower-income areas. In order to assess
the mismatch between the demand and supply for schooling, we use the sum
of students in schools in each area (the demand), and the sum of
individuals within the school grade age-group living in each area (the
supply) in the following competitive accessibility measure, first
proposed by ({\textbf{???}}) and adapted by El-Geneidy et al (2015):

\[ CA_{i}^M= \sum_{j=1}^n\frac{P_{j}-f(C_{ij})}{\sum_{k=1}^n(Y_{k}-f(C_{kj}))}\]

Where the numerator discounts the number of pupils in area \(j\)
\(P_{j}\) by how far area \(i\) is from area \(j\) using the same
function as before, and the denominator discounts the number of students
(young adults) living in zone \(k\) \(Y_{k}\) by how far they are from
area \(j\). In this way, the discounted number of study places at each
area is divided by the potential students available to fill those
places, and then summed in order to obtain a single accessibility
measure for each area \(i\).

In order to assess equality in access to schooling, we stratify our
population of students in two student groups (we could do it for more
income groups): the students living with a head of household who earns 1
minimum wage or less, and those living with a head of household who
earns more than 1 minimum wage. The accessibility measures are based on
student's travel times by different modes using data from the 2007
Origin-Destination survey. Alternatively, for the case of public
transport, wet use actual transit times in 2015 for comparison (bilhete
unico data?).

\section{Results}\label{results}

\begin{itemize}
\tightlist
\item
  Descriptive statistics about modal choice and travel time by AEPs and
  level of income
\item
  Spatial Distribution of Schools and Transit-Dependent Students at the
  AEP Level
\end{itemize}

\section{Discussion}\label{discussion}

\section{Conclusion}\label{conclusion}

\section{References}\label{references}

\hyperdef{}{ref-allen1993accesibility}{\label{ref-allen1993accesibility}}
Allen, W Bruce, Dong Liu, and Scott Singer. 1993. ``Accesibility
Measures of US Metropolitan Areas.'' \emph{Transportation Research Part
B: Methodological} 27 (6). Elsevier: 439--49.

\hyperdef{}{ref-boisjoly2016}{\label{ref-boisjoly2016}}
Boisjoly G., El-Geneidy A. 2016. ``Is Complexity Better? Assessing
Accessibility Measures That Account for Daily Fluctuations in Transit
and Jobs Availability.'' 95th Annual Meeting of the Transportation
Research Board, Washington D.C., USA.

\hyperdef{}{ref-hernandez2015mobilities}{\label{ref-hernandez2015mobilities}}
Hernandez, Daniel Oviedo, and Helena Titheridge. 2015. ``Mobilities of
the Periphery: Informality, Access and Social Exclusion in the Urban
Fringe in Colombia.'' \emph{Journal of Transport Geography}. Elsevier.

\hyperdef{}{ref-ingram1971concept}{\label{ref-ingram1971concept}}
Ingram, David R. 1971. ``The Concept of Accessibility: A Search for an
Operational Form.'' \emph{Regional Studies} 5 (2). Taylor \& Francis:
101--7.

\hyperdef{}{ref-maia2016access}{\label{ref-maia2016access}}
Maia, Maria Leonor, Karen Lucas, Geraldo Marinho, Enilson Santos, and
Jessica Helena de Lima. 2016. ``Access to the Brazilian City---From the
Perspectives of Low-Income Residents in Recife.'' \emph{Journal of
Transport Geography}. Elsevier.

\hyperdef{}{ref-de2015changes}{\label{ref-de2015changes}}
Sá, Thiago Hérick de, Leandro Martin Totaro Garcia, Grégore Iven Mielke,
Fabiana Maluf Rabacow, and Leandro Fórnias Machado de Rezende. 2015.
``Changes in Travel to School Patterns Among Children and Adolescents in
the São Paulo Metropolitan Area, Brazil, 1997--2007.'' \emph{Journal of
Transport \& Health} 2 (2). Elsevier: 143--50.

\end{document}
